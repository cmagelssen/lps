\documentclass{article}


\usepackage{graphicx} % Required for inserting images
\usepackage{biblatex} %Imports biblatex package
\usepackage{booktabs} % Required for tables
\usepackage{longtable}
\usepackage{float}
\usepackage{rotating} % Required for sideways table

\addbibresource{references.bib}


\title{Learning to pump in slalom}
\author{Christian Magelssen}
\date{February 2024}


\begin{document}


\section{Introduction}
Unleashing excellence in sports requires selection of good strategies to help athletes overcome plateaus and attain success. The regulations in alpine ski racing allow a broad range of such strategies, as long as the skier passes the correct side of the gates that make up a slalom course. Due to the redundancy of this sporting task, there has been great engagement among skiers, coaches and scientists in uncovering the underlying strategies for effective ski racing to use this knowledge to train the current and next generation of skiers \cite{joubertHowSkiNew1967, joubertSkiArtTechnique1978, mullerAnalysisBiomechanicalCharacteristics1994, lemasterSkierEdge1999, lemasterUltimateSkiing2010}.

One section of a slalom course where performance has not yet asymptoted for many skiers is the flat section of slalom course. In this section of the course, the component of the gravity vector that propels the skiers down the course is low, which changes what an effective strategy is for this section. One way to ski effectively in this section is to use the "pumping to increase velocity" technique, which is described as extending the body around the. This effect has been observed in many skilled and elite skiers in various disciplines and course.

In a previous study we conducted a large five-day learning experiment on the contextual interference effect in an indoor skiing hall where skilled ski racers trained the pumping technique. To improve this technique we gave the skiers a short introduction of the pumping technique supported by theoretical explanation of its mechanics as well as videos with elite skiers demonstrating this skill. Although we did not find convincing evidence for the contextual interference effect, the skiers improved their flat section massively, suggesting that the pumping technique can be an important skill.

Before and after the skiers underwent the intervention, we recorded the skiers positions with local positioning system while they skied the slalom course. This data can be informative to understand more closely how the skiers improved their race times. Our question was if some these... To examine this question we estimated the mean differences in velocity and accelerations using multilevel Generalized Additive Models (GAMs).

\subsection{Results}












\end{document}
